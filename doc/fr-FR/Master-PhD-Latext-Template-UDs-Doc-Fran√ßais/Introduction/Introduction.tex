%\myChapterStar{Titre}{Titre court}{Ajouter à la table des matières? (false|true|chapter|section|subsection|subsubsection -chapter par défaut-)}
\myChapterStar{Introduction}{}{true}
%\myMinitoc{Profondeur de la minitoc (section|subsection|subsubsection)}{Titre de la minitoc}
\myMiniToc{}{Contents}

%\mySectionStar{Titre}{Titre court}{Ajouter à la table des matières? (false|true|chapter|section|subsection|subsubsection -section par défaut-)}
\mySectionStar{Le contexte du travail}{}{true}
Le moyen de communication a toujours été un besoin et une nécessité pour l'être humain. Le fait de chercher à échanger le plus grand nombre d'informations dans un délai plus court et parfois à des distances plus grandes constitue un grand défi que les inventeurs et les chercheurs se sont décidés de relever en faisant progresser des techniques de la communication et de l'information : de l'invention de l'écriture à la création des réseaux organisés de télécommunication, installés depuis plus d'un siècle dans notre vie quotidienne.
L'internet, incontournable que ce soit dans notre vie privée ou sociale, offre divers services parmi lesquels la messagerie électronique, le web, etc.
Le web, nom anglais signifiant « Toile » et contraction de Wold Wide Web (toile mondiale), est une possibilité offerte sur le réseau internet de naviguer entre des documents (pages web) via des liens hypertextes. Ainsi, le web offre une opportunité à toute organisation d'automatiser son système d'échange d'information et de communication avec ses différents partenaires.
Les possibilités d'accès à distance aux applications de gestion via internet ont permis à ces organisations de rendre plus rentables leurs activités principales et plus rapide le traitement de grandes quantités d'informations.
Pour ce faire, ces organisations sont appelées à observer, analyser, comprendre leur système pour concevoir des applications web qui correspondent à leurs besoins.
En informatique, lorsqu'un ensemble de codes permet aux utilisateurs de réaliser des tâches spécifiques dans le cadre d'une activité d'un domaine spécifique, ces codes sont qualifiés d'application. Alors, si ce sont des documents web, lorsque ceux-ci sont composés de manière structurée pour permettre aux utilisateurs d'accomplir des tâches relatives à une activité, on parlera d'application web. Sachant que chaque organisation a toujours des problèmes spécifiques qui lui sont propres, exigeants une étude bien appropriée, dans le cadre de notre travail nous allons faire une étude de mise au point d'une application web mobile de Mise en place et sécurisation d’une plateforme web mobile permettant aux étudiants de consulter leur résultat par ECUE  

\mySubSectionStar{Les documents structurés}{}{true}
gjfjh...

\mySubSectionStar{L'édition coopérative des documents structurés}{}{true}
Le processus \Citep{badouelTchoupeCmcs, theseTchoupe} consiste à faire une étude de mise au point d'une application web mobile de Mise en place et sécurisation d’une plateforme web mobile permettant aux étudiants de consulter leur résultat par ECUE  

\mySectionStar{La problématique étudiée}{}{true}
La République du Bénin a lancé un programme ambitieux de développement de l’économie numérique visant à positionner le pays comme la référence en matière de plateforme de services numériques en Afrique de l’Ouest à l’horizon de 2021 et de faire des Technologies de l’Information et de la Communication le principal levier de son développement socio – économique.
C’est dans ce contexte qu’a été identifié le besoin pour les trois (03) ordres de l’enseignement à savoir :( Primaire, Secondaire et Supérieur) la mise en place d’un projet. Le projet e-Education est ancré dans le PAG (Programme d’Action du Gouvernement)  Pilier 2 | Engager la transformation structurelle de l’économie  et l’amélioration des performances de l’éducation.
Dans le but de se conformer aux résultats issus des missions d’audits qui révèle que la plateforme de gestion des inscriptions de l’Université Nationale d’Agriculture commence par montrer des signes de faiblesse face aux nouvelles normes de l’internet en perpétuelle évolution, quelques changements s’imposent  pour  une meilleure fluidité des données, afin de profiter pleinement du potentiel disponible grâce à l'utilisation du serveur personnel de l'UNA ( Université Nationale d’Agriculture).
Pour arriver à pallier à ces différents problèmes et à automatiser le contrôle des notes, nous préconisons la mise au point d'une application web d’inscription avec plusieurs fonctionnalités. C’est dans ce contexte que nous avons proposé  le sujet : « Mise en place et sécurisation d’une plateforme web mobile permettant aux étudiants de consulter leur résultat par ECUE (Eléments Constitutifs d’Unité d’Enseignement)» 
. Pour y parvenir nous nous posons les questions suivantes :
§ Comment pouvons-nous arriver à modéliser le contrôle des notes des étudiants ?
§ Quelle application convient le mieux à cette gestion ?	
\mySectionStar{Hypothèse}{}{true}
Etant donné que la problématique a déjà été posée, il est important d'y assimiler quelques réponses provisoires, les quelles réponses pourront être confirmées ou infirmées à la fin de ce travail.
Ainsi, vu notre sujet, nous pouvons envisager les possibilités suivantes :
• Utiliser le langage UML avec la méthodologie UP pour bien modéliser le contrôle des moyennes par ECUE de tous les étudiants.
• Pour ce suivi, nous avons trouvé que l'application qui convient est une application web mobile.
\mySectionStar{Hypothèse}{}{true}




\myCleanStarChapterEnd
